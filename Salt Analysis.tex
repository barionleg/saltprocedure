\documentclass[10pt, a4paper]{article}

%\usepackage{mathptmx}
\usepackage{amsmath}
\usepackage{amsfonts}
\usepackage{amsthm}
\usepackage{mathtools}
\usepackage[margin=2cm]{geometry}
\usepackage[utf8]{inputenc}
\usepackage[english]{babel}
 \usepackage{listings}
\usepackage{calligra}
\usepackage[T1]{fontenc}
\usepackage{pdftexcmds}
\usepackage[version=3]{mhchem}
\usepackage{multirow}
\newtheorem{mystate}{Statement}

\setlength{\parindent}{4em}

\newcommand{\HRule}{\rule{\linewidth}{0.5mm}}
%=============================BEGINNING OF DOCUMENT=============================================================
\begin{document}

\pagenumbering{arabic}
\begin{center}
\HRule \\[1pt]
\Huge{\textsc{Systematic Salt Procedure }} \\[2pt]
\normalsize{for XI and XII CBSE Students} \\[3pt]
\HRule \\[1pt]
\end{center}
\normalsize{Prof. Sivasankar {\textsc T}  \hfill Edilebert {\textsc R}} \\[3pt]
\section*{Prelimnary Tests}
\begin{tabular}{| l | p{5.3cm} | p{5cm} | p{5.3cm} |}
\hline
 & \bf{Experiment}& {\bf Observation} & {\bf Inference} \\ \hline

1 & {\bf Noted the colour of the given salt} & (i) Colourless \newline \newline (ii) Blue \newline (iii) Flesh Colour \newline (iv) Green \newline (v) Pink  &  Absence of  \ce{Cu^{2+}}, \ce{Ni^{2+}}, \ce{Co^2+}, \ce{Mn^{2+}} \newline Presence of \ce{Cu^2+} ions \newline Presence of \ce{Mn^2+} ions \newline Presence of \ce{Ni^2+} ions \newline Presence of \ce{Co^2+} ions\\ \hline

2 & {\bf Noted the Smell of the salt :} A pinch of the salt is taken in a watch glass and rubbed with a drop of water& Smell of Vinegar \newline Smell of Ammonia \newline No Characteristic Smell& Presence of \ce{CH_3COO^- } ions \newline Presence of \ce{NH_4^+} ions \newline Absence of \ce{CH_3COO^- } and \ce{NH_4^+} ions  \\ \hline

3 & {\bf Solubility in Water :} A pinch of salt is shaken well with water& Soluble and No precipitate with Sodium Carbonate Solution \vspace{3pt}\newline Soluble and precipitate with Sodium Carbonate Solution& Presence of \ce{NH_4^+} \newline \vspace{3pt}\newline Absence of \ce{NH_4^+}\\ \hline

4 & {\bf Action of Heat : } About 0.5\,g of the salt is heated in a {\em dry }test tube &
 Colourless and Odourless gas which turns lime water {\em milky} \newline Vapours with Vinegar Smell \newline Reddish Brown Vapours \newline Violet Vapours \newline Colourless Gas with the smell of Ammonia \newline White Residue \newline No Characteristic Change & Presence of Carbonates \newline \newline Presence of Acetates \newline Presence of Nitrates \newline Presence of Iodides \newline Presence of Ammonia \newline \newline Presence of Zinc \newline Absence of \ce{CO3 ^{2-} }, \ce{CH3COO-}, \ce{NO3^2-}, \ce{I-}, \ce{NH4^+}, \ce{Zn^2+}\\ \hline 

5 & {\bf Flame Test : } A paste of the salt and conc.HCl is prepared, and the flame test is performed & Green Coloured Flame \newline Crimson Red Coloured Flame \newline Brick Red Coloured Flame \newline Bright Bluish Green Fame& Presence of Barium \newline Presence of Strontium \newline Presence of Calcium \newline Presence of Copper\\ \hline
\end{tabular}

\newpage
\section*{Analysis of Acid Radicals}
\begin{tabular}{| l | p{5cm} | p{5cm} | p{5cm} |}
\hline
 & \bf{Experiment}& {\bf Observation} & {\bf Inference} \\ \hline
1 & {\bf Test with dil. Hydrochloric Acid : } To a little of the salt, 1\,mL of dil.HCl is aded& & \\ \hline
2 & & & \\ \hline
3 & & & \\ \hline
4 & & & \\ \hline
5 & & & \\ \hline
6 & & & \\ \hline
\end{tabular}
\newpage
\subsection*{Conformation of Acid Radicals}
\begin{tabular}{| l | p{5cm} | p{5cm} | p{5cm} |}
\hline
 & \bf{Experiment}& {\bf Observation} & {\bf Inference} \\ \hline
1 & & & \\ \hline
2 & & & \\ \hline
3 & & & \\ \hline
4 & & & \\ \hline
5 & & & \\ \hline
6 & & & \\ \hline
\end{tabular}
 
\newpage
\section*{Analysis of Basic Radicals}
\subsection*{1) Zero$^\mathbf{th}$ Group Analysis}
\begin{tabular}{|   p{5cm} | p{5cm} | p{5cm} |}
\hline
 \bf{Experiment}& {\bf Observation} & {\bf Inference} \\ \hline
 & & Presence of Ammonia (NH$_4 ^+$) \\ \hline
\end{tabular}

\subsection*{Group Analysis}

\begin{tabular}{| p{2cm} | p{2cm} | p{2cm} | p{2cm} | p{2cm} | p{2cm} | p{1.9cm} |}
\hline
\multicolumn{7}{| p{2cm} |}{ Group 1 test... } \\ \hline
observation... Presence of Group I ions & \multicolumn{6}{| p{2cm} |}{ if group 1 test fails, Group 2 test... } \\ \cline{2-7}
 & observation 2... & \multicolumn{5}{| p{2cm} |}{ if group 2 test fails, Group 3 test... } \\ \cline{3-7}
 & & observation 3... & \multicolumn{4}{| p{2cm} |}{ if group 3 test fails, Group 4 test... } \\ \cline{4-7}
 & & & observation 4... & \multicolumn{3}{| p{2cm} |}{ if group 4 test fails, Group 5 test... } \\ \cline{5-7}
 & & & & observation 5... & \multicolumn{2}{| p{2cm} |}{ if group 5 test fails, Group 6 test... } \\ \cline{6-7}
 & & & & & observation 6... & Presence of Group 6 ions \\ \hline
\end{tabular}

\newpage

\subsection*{Conformation of Basic Radicals}
\begin{tabular}{| l | p{5cm} | p{5cm} | p{5cm} |}
\hline
 & \bf{Experiment}& {\bf Observation} & {\bf Inference} \\ \hline
1 & & & \\ \hline
2 & & & \\ \hline
3 & & & \\ \hline
4 & & & \\ \hline
5 & & & \\ \hline
6 & & & \\ \hline
\end{tabular}

\section*{Result}
\begin{itemize}
\item The Acid Radical is found to be 
\item The Basic Radical is found to be
\end{itemize}
Therefore, The given salt is found to be \\


{\centering 
\HRule} 
\end{document}